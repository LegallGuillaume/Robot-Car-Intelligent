Valeur qui peut etre Négative !! /!\textbackslash{} A\-T\-T\-E\-N\-T\-I\-O\-N, P\-R\-E\-N\-D\-R\-E L\-A V\-A\-L\-E\-U\-R A\-S\-S\-O\-C\-I\-E\-R

\subsection*{\#\# Nombres }

\begin{quotation}
Valeur signé\-:

\end{quotation}


\begin{quotation}
int8\-\_\-t = valeur de -\/128 à 127

\end{quotation}


\begin{quotation}
int16\-\_\-t = valeur de -\/32 768 à 32 767

\end{quotation}


\begin{quotation}
int32\-\_\-t = valeur de -\/2 147 483 648 à 2 147 483 647

\end{quotation}


\begin{quotation}
Valeur non signé\-:

\end{quotation}


\begin{quotation}
uint8\-\_\-t = valeur de 0 à 255

\end{quotation}


\begin{quotation}
uint16\-\_\-t = valeur de 0 à 65 535

\end{quotation}


\begin{quotation}
uint32\-\_\-t = valeur de 0 à 4 294 967 295

\end{quotation}


\subsection*{Vector}






\begin{DoxyItemize}
\item 10 elements dans le tableau. ```cpp \#include $<$vector$>$ std\-::vector$<$int8\-\_\-t$>$ $\ast$ptr = new std\-::vector$<$int8\-\_\-t$>$(); std\-::vector$<$int8\-\_\-t$>$ non\-\_\-ptr;
\end{DoxyItemize}

ptr-\/$>$reserve(10); non\-\_\-ptr.\-reserve(10);

delete ptr; ```

\subsection*{Allocation dynamique}






\begin{DoxyItemize}
\item Création tableau à 2 dimensions de 20x20.
\item -\/128 $<$ 20 $<$ 127 --$>$ int8\-\_\-t
\end{DoxyItemize}

```cpp int8\-\_\-t $\ast$$\ast$tableau; uint8\-\_\-t allocmem = 0; try\{ tableau = new int8\-\_\-t $\ast$ \mbox{[} 20 \mbox{]}; std\-::fill\-\_\-n( tableau, 20, static\-\_\-cast$<$int8\-\_\-t$\ast$$>$(0)); /$\ast$permet d'initialiser le tableau à 0 for(allocmem = 0; allocmem $<$ 20; ++allocmem) \{ tableau\mbox{[} allocmem \mbox{]} = new int8\-\_\-t\mbox{[} 20 \mbox{]}; \} \} catch( const std\-::bad\-\_\-alloc \&) \{ /$\ast$si l'allocation ne s'est pas bien passé, alors on libère l'espace utilisé for( uint8\-\_\-t i = 0; i $<$ allocmem; ++i )\{ delete \mbox{[}\mbox{]} tableau\mbox{[} i \mbox{]}; \} delete \mbox{[}\mbox{]} tableau; throw; \} ```


\begin{DoxyItemize}
\item Destruction du tableau 20x20
\end{DoxyItemize}

```cpp for(int8\-\_\-t i=0; i$<$20; i++)\{ delete \mbox{[}\mbox{]} tableau\mbox{[}i\mbox{]}; tableau\mbox{[}i\mbox{]} = nullptr; \} delete \mbox{[}\mbox{]}tableau; tableau = nullptr; ```

\subsection*{\#\# Pair }


\begin{DoxyItemize}
\item position x3 \& y5
\end{DoxyItemize}

```cpp \#include $<$utility$>$

std\-::pair$<$int8\-\_\-t, int8\-\_\-t$>$ position; position = std\-::make\-\_\-pair(3, 5);

std\-::pair$<$int8\-\_\-t, int8\-\_\-t$>$ $\ast$pair; std\-::pair$<$int,int$>$ pt; pt = std\-::make\-\_\-pair(3, 5); paire = 

printf(\char`\"{}\-Valeur de x\-: \%d, y\-: \%d\char`\"{}, pair-\/$>$first, position.\-second);

delete pair; pair = nullptr; ```

{\ttfamily Valeur de x\-: 3, y\-: 5} 